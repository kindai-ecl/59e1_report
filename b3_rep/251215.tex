\documentclass[10pt,twocolumn, a4j]{jsarticle}
\usepackage{setspace}
\usepackage{color}
\usepackage[dvipdfmx]{graphicx}
\usepackage{amsmath}
\usepackage{algorithm}
\usepackage{algorithmic}

\setlength{\oddsidemargin}{-12mm}
\setlength{\topmargin}{-16mm}
\setlength{\textwidth}{181mm}
\setlength{\columnsep}{8mm}

\setlength{\textheight}{248mm}
\renewcommand{\baselinestretch}{.856}

\pagestyle{empty}

\makeatletter
\def\@biblabel#1{#1)}
\def\@cite#1{#1)}
\makeatother

\begin{document}

\twocolumn[
{\LARGE
\begin{center}
ECサイトにおけるJPYCを用いた分散型エスクロー決済フローの設計と実装
\end{center}
}

\vspace{-2mm}
\begin{flushright}
電子商取引研究室\hspace{1.5zw}呉竹 櫂人\hspace{1.5zw}
\end{flushright}
\vspace{3mm}
]

\renewcommand{\thesection}{\arabic{section} .}

\section{序論}

\subsection{背景}
国際的なEC取引では、支払いと商品引き渡しの非同時性からトラブルが発生しやすい。中央集権型エスクローは有効であるが、手数料、国・規約による制限、再利用性の低さが課題である。また、円と同等の価値を持つトークン(JPYC)を用いることで、ブロックチェーン上で円建て取引が可能となったが、既存のEC決済との統合には技術的課題が残る。

\subsection{問題点}
従来の暗号資産決済における主な問題点は以下の通りである。
\begin{itemize}
\item 誤送金やフィッシング詐欺のリスク
\item PC購入時のスマートフォンウォレット決済の設計不在
\item 中央事業者への資金管理の依存
\item 手動送金方式による操作の煩雑さ
\end{itemize}

\subsection{提案手法}
本研究では、JPYCを用いた分散型エスクロー決済フローを設計し、スマートコントラクトにより資金のロック・解放・返金を制御する仕組みを提案する。これにより、中央管理者に依存せず、低手数料かつ誰でも利用・検証可能な国際送金・決済構造をEC取引において実現する。特に、決済を「サービス」ではなく「プロトコル」として設計することで、誤送金耐性、再利用性、透明性の向上を目指す。

\subsection{実験・評価内容}
提案システムの有効性を検証するため、以下の項目について従来手法と定量的に比較評価する。
\begin{itemize}
\item 操作量(クリック数・署名回数・画面遷移数)の測定
\item 手入力項目数(アドレス・金額・チェーン選択)の比較
\item 誤送金テストケースの成功/失敗率
\item 入金検知までの時間(ブロック確定→注文更新)
\item deposit/release/refundに要するガス代
\item 脅威モデル(PC改ざん、フィッシング等)に対する耐性評価
\end{itemize}

\section{JPYC対応エスクロースマートコントラクトの設計}
(今後記述予定)

\section{PC購入・スマホウォレット支払い決済フローの設計}
(今後記述予定)

\section{EC注文とオンチェーン取引の紐付けアルゴリズム}
(今後記述予定)

\section{従来方式との比較評価}
(今後記述予定)

\section{今後の計画}
(今後記述予定)

\end{document}
