\documentclass[10pt,twocolumn, a4j]{jsarticle}
\usepackage{setspace}
\usepackage{color}
\usepackage[dvipdfmx]{graphicx}
\usepackage{amsmath}
\usepackage{algorithm}
\usepackage{algorithmic}

\setlength{\oddsidemargin}{-12mm}
\setlength{\topmargin}{-16mm}
\setlength{\textwidth}{181mm}
\setlength{\columnsep}{8mm}

\setlength{\textheight}{248mm}
\renewcommand{\baselinestretch}{.856}

\pagestyle{empty}

\makeatletter
\def\@biblabel#1{#1)}
\def\@cite#1{#1)}
\makeatother

\begin{document}

\twocolumn[
{\LARGE
\begin{center}
分散台帳技術を用いたヘッドレスECサイトにおけるトークン決済システムの実装と評価
\end{center}
}

\vspace{-2mm}
\begin{flushright}
電子商取引研究室\hspace{1.5zw}呉竹 櫂人\hspace{1.5zw}
\end{flushright}
\vspace{3mm}
]

\renewcommand{\thesection}{\arabic{section} .}

\section{序論}

\subsection{背景}
近年、Eコマース市場は急速に拡大しており、決済手段の多様化が求められている。特に、ブロックチェーン技術を基盤とした暗号資産やトークンによる決済が注目されている。一方、従来のECプラットフォーム(EC-CUBE、WooCommerceなど)は、フロントエンドとバックエンドが密結合されており、新しい決済手段の導入や柔軟なシステム拡張が困難である。

\subsection{問題点}
従来のECサイトにおける主な問題点は以下の通りである。
\begin{itemize}
\item モノリシックなアーキテクチャによる拡張性の低さ
\item トークン決済などの新しい決済手段への対応の遅れ
\item フロントエンドとバックエンドの技術選択の制約
\item セキュリティリスクへの対応の複雑さ
\end{itemize}

\subsection{提案手法}
本研究では、ヘッドレスコマースアーキテクチャとブロックチェーン技術を組み合わせた新しいECサイトシステムを提案する。具体的には、WebAPI化されたバックエンドと、分散台帳技術を用いたトークン決済機能を実装する。また、OWASP Top 10に基づくセキュリティチェックと対策を行う。

\subsection{実験・評価内容}
提案システムの有効性を検証するため、以下の実験・評価を行う。
\begin{itemize}
\item トークン決済の処理速度と成功率の測定
\item 従来型ECサイトとの比較評価
\item セキュリティ脆弱性診断の実施
\end{itemize}

\section{ヘッドレスコマースシステムの設計と実装}
(今後記述予定)

\section{今後の計画}
(今後記述予定)

\nocite{*}
\bibliography{251207}
\bibliographystyle{junsrt}

\end{document}
