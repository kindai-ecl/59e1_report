\documentclass[10pt,twocolumn, a4j]{jsarticle}
\usepackage{setspace}
\usepackage{color}
\usepackage[dvipdfmx]{graphicx}
\usepackage{amsmath}
\usepackage{url}

\setlength{\oddsidemargin}{-12mm}
\setlength{\topmargin}{-16mm}
\setlength{\textwidth}{181mm}
\setlength{\columnsep}{8mm}

\setlength{\textheight}{248mm}
\renewcommand{\baselinestretch}{.856}

\pagestyle{empty}

\makeatletter
\def\@biblabel#1{*#1)}
\def\@cite#1{*#1)}
\makeatother

\begin{document}

\twocolumn[
{\LARGE
\begin{center}
ECサイトにおけるJPYCを用いた分散型エスクロー決済フローの設計と実装
\end{center}
}

\vspace{-2mm}
\begin{flushright}
近畿大学 情報学部 情報学科\hspace{1.5zw}呉竹 櫂人\hspace{1.5zw}
\end{flushright}
\vspace{3mm}
]

\renewcommand{\thesection}{\arabic{section} .}

\section{序論}

ECサイトで買い物をする時、お金を払ってから商品が届くまでに時間がかかる。この間にトラブルが起きる可能性がある。今までは運営会社がお金を預かる方法が使われてきたが、手数料が高く、運営会社に頼るしかない問題がある\cite{jpyc2021}。本研究では、日本円と同じ価値を持つJPYCという暗号資産を使い、運営会社に頼らない安全な支払い方法を作ることを目的とする。PCで買い物をして、スマートフォンで安全に支払いができる仕組みを作る。

\section{提案方式}

\subsection{お金を預かる仕組み}
買い手はJPYCを一時的に預ける。注文番号で取引を区別し、「最初」「預けた」「渡した」「返した」の4つの状態で管理する。商品が無事届いたら売り手にお金を渡す。期限を過ぎたら買い手にお金を返す。この仕組みで、お金を払ったのに商品が届かない時でも、ちゃんとお金が戻ってくる。

\subsection{支払いの流れ}
PCの画面に、注文番号・金額・売り手の情報が入ったQRコードを表示する。買い手はスマートフォンでQRコードを読み取り、送り先と金額を確認してから送金する。大事な秘密の情報(秘密鍵)はスマートフォンの中だけに保管し、PCには入力しない。手で入力する項目を減らすことで、アドレスや金額を間違えて入力するミスを防ぐ。

\section{評価方法}

この研究では、「コスト」と「資金ロック率」の2つを測る。比べる方法は、(a)普通の送金と(b)提案する方法の2つとする。

\subsection{コスト}
1回の注文でかかる手数料を円で計算して比べる。普通の送金は送金1回分の手数料とする。提案する方法は、お金を預ける時と渡す時の2回分の手数料を合計する。それぞれの手数料を記録して、その時の円の価値で計算する。ネットワークの混み具合で手数料が変わるため、真ん中の値と高い方の値を示す。

\subsection{資金ロック率}
お金を預けた注文の中で、期限を過ぎてもお金が渡されず、返ってもこない割合を資金ロック率と呼ぶ。もしお金が動かなくなった時は、注文番号、状態、記録、理由を残して原因を調べる。資金ロック率が0に近ければ、この仕組みがちゃんと動くことを示せる。

\section{今後の計画}

本研究の実施スケジュールを表1に示す。

\begin{table}[h]
\centering
\caption{研究スケジュール}
\begin{tabular}{|l|l|}
\hline
時期 & 作業内容 \\ \hline
4月 & 要件分析・関連研究調査 \\ \hline
5月 & 決済フロー・仕様設計 \\ \hline
6月 & スマートコントラクト実装 \\ \hline
7月 & EC側プロトタイプ実装 \\ \hline
8月 & テスト・予備実験 \\ \hline
9月 & 実証実験・評価 \\ \hline
10月-11月 & 論文執筆・まとめ \\ \hline
\end{tabular}
\end{table}

\subsection{必要機材}
\begin{itemize}
\item イーサリアム(テスト用のJPYC等を受け取るため、前期中に準備)
\end{itemize}

\section{結論}

本研究では、JPYCを使った安全な支払い方法と、PCでQRコードを表示してスマートフォンで支払う流れを提案する。評価は「コスト」と「資金ロック率」の2つに絞り、実際に使えるかどうかと信頼できるかを数字で示す。今後は実際に作って動かし、手数料がどのくらいかかるか、お金が動かなくなる条件は何かを整理する。

\bibliographystyle{junsrt}
\bibliography{260120}

\end{document}
